%%%\chapter{INTRODUCTION}
%%%
%%%% \pagestyle{plain}
%%%
%%%\label{introduction}
%%%
%%%\section{Charged Higgs studies}
%%%
%%%\subsection{production}
%%%
%%%\begin{outline}
%%%\1 produced mainly through top quark decays (top->bottom, Higgs-plus)
%%%	\2 production cross section heavily influenced by top quark production cross section
%%%\1 top quarks obtained through SM ttbar production at LHC
%%%	\2 also through single top production (give diagrams for both processes)
%%%\1 production classified between high-mass and low-mass
%%%	\2 low mass is from ttbar, one top quark decaying to a bottom quark and forming a charged higgs. Another quark may 
%%%	decay similarly emitting a W instead
%%%	\2 high mass has two contributing diagrams
%%%		\3 bbar quark absorbs a gluon and decays to a tbar in association with a higgs (5 flavor scheme)
%%%		\3 bbbar and ttbar production from two gluons ( dont understand this diagram yet )
%%%		\3 High mass because rather than being bounded by the top mass, they are from a collision between top and bottom quarks 
%%%\end{outline}
%%%
%%%\subsection{decay channel}
%%%\begin{outline}
%%%\1 have to define a tanBeta variable
%%%\1 In all regions of tanBeta the higgs to taunu is very significant
%%%	\2 for low tanBeta it is dominant
%%%\1 this is the choice for the decay channel
%%%	\2 limits at 7 TeV BR(Higgs to taunu) was assummed at 100\%. This was for low tanBeta
%%%\1 Signal is then characterised by the following:
%%%	\2 low mass:
%%%		\3 two b-jets
%%%		\3 hadronic tau
%%%		\3 missing et
%%%		\3 hadronic W (decay branching fraction higher than leptonic)
%%%	\2 high mass is similar, but b-jet multiplicity is forgiven to allow for 5 flavor scheme diagram
%%%\end{outline}
%%%
%%%%\subsection{qcd}
%%%%
%%%%\begin{outline}
%%%%\1 emmision or absorption of gluons by quarks changes the color of the quark
%%%%	\1 but color must be conserved
%%%%\1 think of color as follows:
%%%%	\2 red -> vector x
%%%%	\2 green -> vector y
%%%%	\2 blue -> vector z
%%%%\1 gluons also have color charge
%%%%	\2 compare to a photon with zero charge
%%%%	\2 color of gluons is a combination of two colors (color + anticolor)
%%%%	\2 a red-green gluon changes a red quark into a green quark
%%%%\end{outline}
%%%
%%%%  vkp
