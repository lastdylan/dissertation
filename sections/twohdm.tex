\subsection{Two-Higgs-Doublet Model}
\par One of the models that predict a family of Higgs bosons is the
Two-Higgs-Doublet Model (2HDM)~\cite{PhysRevD.8.1226}. 
The phenomenology of this model is categorized into two classes: type-I and type-II. The 
type-II model corresponds to the hMSSM. MSSM is built on two Higgs 
doublet fields 

\begin{equation}
\begin{aligned}
\phi_1 = \begin{pmatrix}\phi_1^+ \\ \phi_1^0 \end{pmatrix} &\text{  with VEV } & \langle\phi_1\rangle = \frac{v_1}{\sqrt{2}}, & \text{ and, } \\ 
\phi_2 = \begin{pmatrix}\phi_2^+ \\ \phi_2^0 \end{pmatrix} &\text{  with VEV } & \langle\phi_2\rangle = \frac{v_2e^{i\theta}}{\sqrt{2}}. &  
\end{aligned}
\label{eq:fields}
\end{equation}

In type-I 2HDM all fermions couple to one of the Higgs doublets, while in type-II up-type quarks couple to 
a different doublet from down-type quarks. Following the strategy used to construct 
SSB in the preceding section, the doublet fields can be generalized as 

\begin{equation}
\begin{aligned}
\phi_1 = \begin{pmatrix}\phi_1^+ \\ \frac{h_1 + v_1 + ig_1}{\sqrt{2}} \end{pmatrix}, \\ 
\phi_2 = \begin{pmatrix}\phi_2^+ \\ \frac{h_2 + v_2e^{i\theta} + ig_2}{\sqrt{2}} \end{pmatrix}.
\end{aligned}
\label{eq:fieldsGen}
\end{equation}

The total Lagrangian for the system is  
$\mathcal{L} = \mathcal{L}_{kin} + \mathcal{L}_{pot} + \mathcal{L}_{Yukawa}$, which is a sum 
of kinetic, potential and Yukawa terms.  $\mathcal{L}$ is required to conserve both charge and charge-parity (CP)
symmetries, just like the Lagrangian in the SM. The potential term,  $\mathcal{L}_{pot} = \braket{\phi^\dagger|V|\phi}$ 
is built by 

\begin{equation}
V = -\mu_1^2\hat{A}-\mu_2^2\hat{B}+h_1\hat{A}^2+h_2\hat{B}^2+h_3(\hat{C}^2+\hat{D}^2)+h_5\hat{A}\hat{B}
\end{equation}

where 

\begin{equation*}
\hat{A} \equiv \phi_1^\dagger\phi_1, \hat{B}\equiv\phi_2^\dagger\phi_2,\hat{C}\equiv\Re(\phi_1^\dagger\phi_2), 
\hat{D}\equiv\Im(\phi_1^\dagger\phi_2).
\end{equation*}

This potential is designed to conserve both charge and parity symmetries as well.
 The $\mu_i,h_i$ are free terms that eventually 
define the Higgs masses. From these parameters the Higgs bosons $H,h,A$ and 
$(H^{\pm})$ emerge. The studies presented in this text are concerned only with $H^{\pm}$. 
Present discussion is therefore limited to $H^{\pm}$. 
Moreover, since the 2HDM Lagrangian is invariant under charge transformations,  
$H^{\pm}$ is referred to as $H^+$, ignoring charge. 

\par In this setup both doublets can have real 
vacuum expectation values (VEVs). The ratio of these VEVs is known as  
$\tan\beta$. The entire parameter space of this Higgs sector is defined by $\tan\beta$ 
and the mass of $H^{\pm}$, $m_{H^{\pm}}$.    

\par The minimum conditions on $V$ give rise to the following solutions  
 
\begin{equation}
\centering
\begin{aligned}
v_1^2 &= \frac{h_1-h_2\pm Z_1}{2(h_1 - h_+)(h_2 - h_+)} \\
v_2^2 &= \frac{h_2-h_1\pm Z_2}{2(h_1 - h_+)(h_2 - h_+)} \\
Z_1 &= \sqrt{(h_1-h_2)^2 - 4(h_1-h_+)(h_2-h_+)
\left [ (h_+v^2-\mu_1^2)(h_+v^2-\mu_2^2)-\frac{\mu_3^4}{4}\right ]} \\
Z_2 &= \sqrt{(h_1-h_2)^2 - 4(h_2-h_+)(h_1-h_+)
\left [ (h_+v^2-\mu_2^2)(h_+v^2-\mu_1^2)-\frac{\mu_3^4}{4}\right ]} \\
\end{aligned}
\end{equation}

giving rise to the following higgs masses 

\begin{equation}
\begin{aligned}
m_{H^{\pm}}^2 &= -h_3(v_1^2 + v_2^2)+\mu_3^2\frac{v_1^2+v_2^2}{v_1v_2} \\
m_{A^{0}}^2 &= \frac{1}{2}\mu_3^2\frac{v_1^2+v_2^2}{v_1v_2} \\
\end{aligned}
\end{equation}

and 

\begin{equation}
\begin{split}
m_{H^0,h^0}^2 = h_1v_1^2 + h_2v_2^2 + \frac{1}{4}\mu_3^4(\tan\beta+\cot\beta) \\
\pm \sqrt{\left [ h_1v_1^2 - h_2v_2^2 + \frac{1}{4}\mu_3^4(\tan\beta-\cot\beta)\right ]^2  
+ \left (2v_1v_2h_+ - \frac{1}{2}\mu_3^2  \right )^2}. 
\end{split}
\end{equation}

Expanding the potential term of the Lagrangian reveals that $\Hplus$ can interact with 
other Higgs bosons in the sector,  
with a coupling strength dependent on the VEVs of the doublets, which is $\tan\beta$.

\par The kinetic term of the Lagrangian is 

\begin{equation}
\mathcal{L}_{kin} = (D_\mu\phi_1)^\dagger(D_\mu\phi_1) + (D_\mu\phi_2)^\dagger(D_\mu\phi_2)  
\end{equation}

where $D_\mu = \partial_\mu + gL_iW^i_\mu + g^{'}L_4B_\mu$. The $i$ index runs from 1 to 3.  
$L_i$ and $L_4$ are $4\times4$ matrices. Expanding this term reveals that the charged Higgs boson can also 
interact with gauge bosons, with vertices such as $\Zboson H^+H^-$. 

\par Interaction of 2HDM Higgs bosons with fermions is governed by the Yukawa term in the Lagrangian 

\begin{equation}
-\mathcal{L}_{Yukawa} = \eta_{ij}^{D}\overline{Q}_{iL}\phi_1D_{jR} + \varepsilon_{ij}^{U}\overline{Q}_{iL}i\sigma_2\phi_2U_{jR} + \text{leptonic sector} + ...
\label{eq:yk}
\end{equation}

$\eta^0_{ij},\varepsilon^0_{ij}$ are non-diagonal $3\times 3$ matrices and $i,j$ denote quark family 
indices. $\overline{Q}_{iL}$ denote the quark weak isospin left-handed doublets. $D_{jR}, U_{jR}$ denote the 
up-type and down-type weak isospin right-handed singlets respectively. The leptonic sector is similar to 
the quark sector. One Higgs doublet couples to the up-type fermions $(\phi_2)$ and the other couples to the 
down-type fermions. 
