%1. Event generation
% 	- data output ... HepMC
%	- generates prompt decay particles (Z, W, etc) and stores 'stable' ones
%		* unstable particles decay before reaching detector
%		* so, no need to consider detector geometry at this stage
%	- Events with desirable properties may be kept at this stage, otherwise discarded
%3. Records, called 'truth', religiously kept
%	- Generation
%		* incoming/outgoing particles in hard scatter
%		* decay products
%		* simulated or not
%	- Used to evaluate object reconstruction
%
%
%
\par Particles with lifetimes such that they can traverse at least a femtometer from the collision 
point are considered stable. Hadrons from parton hadronization are also considered stable. 
These are run into a simulation of the ATLAS detector to mimic real particles. 

\par Data used to simulate the ATLAS detector is stored in several databases.
This data is made up of details describing physical volumes and material types. 
Physical volumes are used to model sub-detector components, made up of their 
corresponding material types. Ultimately, this 
data comprises a complex arrangement of hundreds of materials and hundreds of thousands of 
physical volumes. For example, calorimeter sub-detectors are made up of over 100 000 physical 
volumes that require about 40~MB or memory to load. The total memory required to load 
the entire ATLAS detector from these databases is almor 300~MB.  

\par Simulation of the detector's response to particles is handled by GEANT4~\cite{geant4}. GEANT4 
is a toolkit that provides models of how particles traverse through several geometries 
and material types. During Run I, GEANT4 version 9.4 was used. Version 9.6 was used 
during Run II. This tookit allows the ability to turn on and off detector components, 
reducing the memory needed to load the detector. It also allows realignments of sub-detectors 
to match simulation conditions to those in specific data runs. This flexibility has enabled several 
standardized detector setups that run faster than loading the whole detector. One of these 
setups that is used in this thesis is ATLFAST2~\cite{PUB-2010-013}; it uses full simulation for the 
Inner Detector and the Muon Spectrometer, but uses a simplified version of the calorimeter 
simulation. 

\par Particle decays, such as photon conversions, are added to the truth record 
at this stage. The detector response, through so-called `hits', is stored in a separate file.   
The hits are inputs to the digitization stage, where the analog signals are converted to 
`digits'. Other simulated $pp$ collisions are overlaid on the primary collision at 
a known rate to mimic pileup effects. This rate would later be adjusted according to the rate 
extracted from data. Detector noise is also added to the event at this stage.  
The event is then passed through the L1 trigger system. Subsequent treatment of simulation data is identical 
to that of real data expect that simulated events that fail the L1 trigger are 
not discarded. The failure is however recorded in the event. 
The output of both simulated data and real data is in the Raw Data Object (RDO) format. This output is 
finally passed through the HLT, and particles are reconstructed from the digitized signals. 
This reconstruction is discussed in the next chapter.  

