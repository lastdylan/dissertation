\par The physics objects used in this analysis are electrons, muons, jets and missing 
transverse energy. Tracks, regardless of what physics object they are associated with, 
are also used. Reconstruction and identification of these objects is discussed in detail 
in Chapter~\ref{obj}. This section summarizes other selection criteria imposed on these 
objects, specific to this analysis. 

\par Track reconstruction is discussed in Section~\ref{sec:idtracks}. Tracks used here are those 
reconstructed from the Inner Detector (ID). Other than satisfying all the track quality criteria 
outlined in the said section, they were required to have left at least 1 hit in the Pixel 
Detector, and 4 hits in the Semi-Conductor Tracker (SCT). Additionally, they were required to have 
$\pt$ of at least 400~\MeV. 

\par Electron reconstruction and identification is discussed in Section~\ref{sec:eleReco}, along with 
the associated corrections and calibrations. The electrons used here were reconstructed from matches between 
calorimeter energy deposits and ID tracks. Since the ID is limited to $|\eta|<2.47$ and the transition region 
between the barrel and end-cap calorimeters ($1.37<|\eta|<1.52$) is a dead region, electron candidates are required 
to not be in the calorimeter transition region while being within $|\eta|<2.47$. So, only central electrons 
were considered. They were also required to pass the very tight likelihood identification selection 
criteria and have $\pt$ of at least 10~\GeV. The electron \pt\ was calibrated and 
corrected as discussed in Section~\ref{sec:eleCorr}. Additional isolation criteria were applied to suppress 
backgrounds from jets that were reconstructed as electrons, both in the calorimeter and tracking 
measurements. As shown in Section~\ref{sec:eleCorr}, such background contamination is energy (or \pt) 
dependent, so these isolation criteria were binned in electron \eT\ and \pt, where \eT was measured 
from the calorimeters and \pt\ was measured from the ID. The variable of interest in defining isolation is 
{\it EtCone30} (or {\it PtCone30}): the \eT\ (or \pt) in a cone of $\Delta R<0.3$ around the electron candidate, excluding 
\eT\ (or \pt) from the electron candidate. The isolation criteria is such that EtCone30 (PtCone30) is less 
than a certain fraction of the electron, \eT\ (\pt). Table~\ref{tab:eleIso} shows this isolation selection criteria 
for electrons. 

\begin{table}[!h]
\begin{center}
\begin{tabular}{|l|cc|}
\hline
           								& Condition                & Isolation Criterion   \\
\hline\hline
					& &  \\
\multirow{3}{*}{Calorimeter}  &  $E^{e}_{\mathrm{T}} < 15$~GeV	   & $\mathrm{EtCone30} / E^{e}_{\mathrm{T}} < 0.20$ \\
													& 	$15 \le E^{e}_{\mathrm{T}} < 20$~GeV & $\mathrm{EtCone30} / E^{e}_{\mathrm{T}} < 0.24$  \\
													& 	$E^{e}_{\mathrm{T}} \ge 20$~GeV & $\mathrm{EtCone30} / E^{e}_{\mathrm{T}} < 0.28$  \\
					& &  \\
\hline
					& &  \\
\multirow{3}{*}{Track}  &  $p^{e}_{\mathrm{T}} < 15$~GeV &  $\mathrm{PtCone30} / p^{e}_{\mathrm{T}} < 0.06$  \\
												&  $15 \le p^{e}_{\mathrm{T}} < 20$~GeV &  $\mathrm{PtCone30} / p^{e}_{\mathrm{T}} < 0.08$ \\
                        & $p^{e}_{\mathrm{T}} \ge 20$~GeV & $\mathrm{PtCone30} / p^{e}_{\mathrm{T}} < 0.10$ \\
					& &  \\
\hline
\end{tabular}
\end{center}
\caption{Electron isolation criteria.}
\label{tab:eleIso}
\end{table}

\par Muon reconstruction and identification is discussed in Section~\ref{sec:muReco}.
In this analysis, combined (CB) muons were used. These use information from both the 
muon spectrometer (MS) and the ID. Apart from the selection criteria outlined in 
Section~\ref{sec:muReco}, these muons were required to have $\pt>10~\GeV$ and be within 
$|\eta|<2.47$ because of the ID pseudorapidity limitation. Tracks from the ID were also required 
to have at least 1 hit in the Pixel Detector and 5 hits in the SCT. Holes in the SCT (see Figure~\ref{fig:mstrackQ})
were not allowed to be more than 2. Isolation criteria similar to that applied on electron 
candidates were also applied on muon candidates, to suppress background from jets 
reconstructed as muons. Table~\ref{tab:muIso} lists the isolation criteria.    

\begin{table}[!h]
\begin{center}
\begin{tabular}{|l|cc|}
\hline
           								& Condition                & Isolation Criterion   \\
\hline\hline
					& &  \\
\multirow{4}{*}{Calorimeter}  & $p^{\mu}_{\mathrm{T}} < 15$~GeV &  $\mathrm{EtCone30} / p^{\mu}_{\mathrm{T}} < 0.06$ \\
													    & $15 \le p^{\mu}_{\mathrm{T}} < 20$~GeV & $\mathrm{EtCone30} / p^{\mu}_{\mathrm{T}} < 0.12$ \\
															& $20 \le p^{\mu}_{\mathrm{T}} < 25$~GeV & $\mathrm{EtCone30} / p^{\mu}_{\mathrm{T}} < 0.18$ \\
												      & $p^{\mu}_{\mathrm{T}} \ge 25$~GeV & $\mathrm{EtCone30} / p^{\mu}_{\mathrm{T}} < 0.30$ \\
					& &  \\
\hline
					& &  \\
\multirow{3}{*}{Track}  & $p^{\mu}_{\mathrm{T}} < 15$~GeV &  $\mathrm{PtCone30} / p^{\mu}_{\mathrm{T}} < 0.06$ \\
												& $15 \le p^{\mu}_{\mathrm{T}} < 20$~GeV & $\mathrm{PtCone30} / p^{\mu}_{\mathrm{T}} < 0.08$ \\
                        & $p^{\mu}_{\mathrm{T}} \ge 20$~GeV & $\mathrm{PtCone30} / p^{\mu}_{\mathrm{T}} < 0.12$ \\
					& &  \\
\hline
\end{tabular}
\end{center}
\caption{Muon isolation criteria.}
\label{tab:muIso}
\end{table}

\par In contrast to electron and muon selection criteria in the charged Higgs boson search, electrons 
and muons in this analysis were not required to satisfy any selection criteria based on the 
impact parameters or their significance. 

\par Jet reconstruction is discussed in Section~\ref{sec:jets}. 
Jets used in this analysis were reconstructed using the anti-kt 
algorithm with $R=0.4$. They were also required to fall within $|\eta|<4.5$ and have at least 
25~\GeV\ in \pt\ to suppress jets from pileup. While energy scale and resolution corrections 
were applied on these jets, all corrections that depend on the position of the primary vertex 
were not applied. Examples of such corrections are the origin correction, JVF and JVT.\footnote{JVT was 
developed for use during Run II anyway.} Reasons for refraining from such corrections will become 
clearer in the following sections.     

\par Reconstruction of missing transverse energy, \met, is discussed in 
Section~\ref{sec:met}. In particular, the \met\ used here reconstructed the soft 
terms in the \met\ from calorimeter energy deposits that were not 
associated to any other physics objects. 
 
