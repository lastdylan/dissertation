\chapter{Introduction}
\label{intro}
\par Driven by the curiosity to understand the universe and the desire to manipulate 
matter in our universe to make lives better for the whole humanity, scientists 
have been trying to understand matter's fundamental building 
blocks and the interactions between them for over a millennium. 
While studies of components of the universe can be traced back 
to the ancient Greek society,\footnote{Thales of Miletus (500-547 B.C) postulated that 
water is the basic substance of the Earth.} the introduction of the Quantum Theory in the early 1900s,
 when the atom was thought to be the fundamental building block of matter, was revolutionary. 
Scattering experiments by Ernest Rutherford, Hans Geiger and Ernest Marsden showed that 
atoms have a positive nucleus, debunking the theory that the atom was fundamental. The first of many proposals  
on the Quantum Theory were put forward by Max Planck and Albert Einstein, suggesting that radiation may be quantized. 
Arthur Compton's scattering experiments confirmed these proposals when he discovered X-rays in 1923.

\par This pattern of seemingly radical theoretical proposals followed by a series of 
experimental verifications was to be repeated over several decades, discovering more particles 
as candidates for fundamental building blocks of matter and forming a new branch of physics -- Particle 
Physics. This culminated to the Standard Model (SM)~\cite{GLASHOW1961579,PhysRevLett.19.1264}, 
a model that classifies fundamental particles into bosons and fermions.
 
\par While the SM summarizes the well known fundamental particles and their forces, it has also been  
used to predict existence of other particles. Its most recent successful prediction is the existence of 
the Higgs boson, which was discovered in 2012 from data collected from proton-proton scattering 
experiments at the Large Hadron Collider~\cite{Bruning:2004ej} (LHC). 
%Other successes include, but are not limited to, the prediction of 
%the \Wpm\ and the \Zboson\ bosons, the gluon, the top and the charm quarks.

\par As with most models, the SM has its failures. One of the most prominent  
is its inability to solve the {\it hierarchy problem}. 
Simply stated, the hierarchy problem encapsulates the uneasiness with which most 
scientists take the fact that the strength of the weak force is 
$10^{24}$ times that of the gravitational force. 
%In the SM, quantum corrections 
%to the square of the mass of the Higgs boson are expected to 
%scale according to the square of the maximum energy accessible to virtual particles in the 
%vacuum. This expectation suggests that the measured mass of the Higgs boson should be comparable to the 
%Planck mass~\footnote{The Planck mass is large -- $10^{19}$ times the mass of the proton.}. The discovery of 
%the Higgs boson at the LHC, and the subsequent precision measurements of its mass, 
%consistently showed that the mass of the Higgs boson is significantly lower than the Planck mass.
%
In the SM, this problem appears in the form of the unexpectedly small size of the Higgs field strength.
One proposed solution is to fine-tune parameters of the SM to moderate this field strength. 
Done by itself, a fine-tuning of 1 part in $10^{30}$ on SM parameters is required. 
Such level of fine-tuning seems a bit contrived. 
Supersymmetric theories attempt to reduce this level of fine-tuning by postulating additional particles with quantum 
numbers supersymmetric to particles in the SM. With these supersymmetric particles, 
the level of fine-tuning is reduced to 1 part in 100. So far none of these supersymmetric particles have 
been observed in scattering experiments.  

\par Several Minimal Supersymmetric extensions to the Standard Model (MSSM) include 
these supersymmetric particles. In these extensions, the SM Higgs boson is a member of a 
family comprising five physical Higgs bosons. No evidence has been observed for the existence of 
any of these Higgs bosons. Part of this thesis discusses work done towards a search for one of these 
Higgs bosons using data collected by scattering protons at the LHC.  

\par This thesis is divided into two parts. First, a search for the charged Higgs boson (\Hpm), 
a member of the MSSM Higgs sector, is described. Second, a search for a particular 
production mode for the SM Higgs boson, called diffraction, is discussed.  
A brief introduction to diffraction is presented in the next few paragraphs, and its link to 
{\it exclusive} processes is explained. Detailed discussions of exclusive processes and the 
theoretical framework underlying MSSM are presented in Chapter~\ref{theory}. The experimental setup, 
simulations, and reconstruction of physics objects are presented in 
Chapters~\ref{expSetup}, ~\ref{sim}, and~\ref{obj}. Chapters~\ref{chargedH} and 
\ref{exclH} discuss searches for the MSSM \Hpm\ and the exclusive SM Higgs bosons, respectively. 

\par The units used in this thesis, while conventional in the particle physics community, may seem 
unusual. Here, natural units are used, where quantities are measured in terms of physical constants 
such as the speed of light $c$ and the Planck constant $\hbar$. To achieve this, $c$ and $\hbar$ are 
normalized such that $c=\hbar=1$. In this system, Einstein's famous energy-mass relation becomes 
$E = m$. So, energy has the same units as mass -- electron-volts (eV). This formalism significantly 
simplifies algebraic equations. In scattering experiments the likelihood of an event happening is quantified by the event's 
cross section, denoted by $\sigma$. Cross sections, being a measure of area, are quantified in units 
of {\it barns}.   

\par As already alluded to earlier, both the SM Higgs boson and the MSSM \Hpm\ are expected to 
be produced at high energies. For the work presented in this thesis, protons were scattered at the LHC at very high energies to 
provide such an environment. While scattering experiments have been conducted since the early 1900s, 
the LHC provides scattering energies at unprecedented scales. For example, in 2015 and 2016, protons were accelerated to 
6.5~\TeV\ before being scattered against protons of the same energy moving in the opposite direction. 
In experiments of this kind, the scattering energy of concern is that which is measured in the rest frame, 
where the total momentum is zero. For beams with momenta $\vec{p}_1$ and $\vec{p}_2$ 
the square of the center of mass energy, $s$, is defined as

\begin{equation}
s = (p_1 + p_2)^2 = (E_1+E_2, \vec{p}_1 + \vec{p}_2)^2
\label{eq:fC}
\end{equation}  

where $p_i$ are relativistic 4-momenta defined as $p_i = (E_i,\vec{p}_i)$.  
In the LHC case $\vec{p}_1=-\vec{p}_2$ and $E_1=E_2$, simplifying Equation~\ref{eq:fC} to 
$s = 4E_1^2$. For $E_1= 6.5$~\TeV the center of mass energy is 13~\TeV. 

\par The advantage of scattering two beams with opposing and equal momenta over scattering 
a high energy beam with a fixed target can be made obvious by further examining Equation~\ref{eq:fC}.
Suppose the fixed target is a proton, with rest mass $m_p$. Suppose also that the high energy beam 
is made up of protons of the same rest mass. The square of the center of mass energy 
for this system becomes 

\begin{equation}
s = (E_1 + m_p, \vec{p}_1 + \vec{0})^2 = E_1^2 + m_p^2 + 2E_1m_p - |\vec{p}_1|^2
= 2m_p^2 + 2E_1m_p. 
\end{equation} 

For $E_1=6.5\TeV$ the center of mass energy is obviously less than in the former case. 
What is more interesting is that for $E_1=13$~\TeV, and $m_p=0.938$~\GeV\ the center of mass 
energy is lower than 0.2~\TeV. So, to reach $\sqs=13$~\TeV, a fixed target proton experiment needs
a proton beam of about 90 000~\TeV!

\par A search for an exclusively produced SM Higgs boson is motivated chiefly by the need 
to perform precision measurements on the Higgs boson in an environment with as little background 
contamination as possible. In diffractive processes protons may be elastically scattered, emerging 
from the collision intact. While in some cases one or both protons may disintegrate, the remnants 
from the disintegration fall in a region outside the detector phase space. In both cases, the 
diffractively produced Higgs boson is in principle isolated from activity other than that from its 
production. Such a {\it clean} mode of production is necessary in order to minimize systematic 
uncertainties when measuring the Higgs mass, spin or coupling strengths to other particles.  
Part of this thesis sets limits to the production cross section of the exclusive SM Higgs boson using data 
collected LHC collisions in 2012.  

