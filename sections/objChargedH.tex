\par Physics objects used in this search are jets, hadronic $\tau$ leptons, electrons, 
muons and missing transverse energy. Reconstruction and identification of these objects 
from detector signals is discussed in detail in Chapter~\ref{obj}. This section summarizes other 
selection criteria imposed on these physics objects, specific to this analysis. 

\par Basic jet reconstruction is discussed in Section~\ref{sec:jets}. Of the  
jets that were reconstructed using the anti-$k_T$ algorithm with $\Delta R=0.4$, only those 
with $\pT>25~\GeV$ and $|\eta|<2.5$ were considered. For jets with $\pT<60~\GeV$, the Jet Vertex 
Tagger (JVT) was used to reduce effects from pileup events. The $\operatorname{b-}$tagging algorithm 
described in Section~\ref{sec:bTag} was applied on each jet that passed this selection criteria, ultimately 
assigning to it a {\it $\operatorname{b-}$tagging score}. The working point for the score used to identify $\operatorname{b-}$tagged 
jets in this analysis corresponds to a 70\% efficiency in identifying $\operatorname{b-}$jets 
in \ttbar events, at a background rejection rate of 400.   

\par Primary reconstruction of hadronic $\tau$ leptons is discussed in Section~\ref{sec:tau}. 
Since every hadronic $\tau$ decay includes a $\nu_\tau$ in its final state, the reconstruction 
was performed only on the visible components of the decay products. The reconstructed object, which 
estimates the $\tau$ lepton, is referred to here as a \tauvis.   
Treatment of $\tauvis$ was separated into 1-prong and 3-prong candidates, corresponding to decays 
to 1 $\pipm$ or 3 $\pipm$ respectively.  A boosted decision 
tree (BDT) was trained to distinguish $\tauvis$ candidates from jets initiated by quarks or gluons. 
Loose and medium working points on the BDT output were used for different purposes in this search. 
The loose working point corresponds to 70\% and 65\% identification efficiencies on the 1-prong and 3-prong 
$\tauvis$ candidates. The medium working point corresponds to 55\% and 40\% identification efficiencies on 
the 1-prong and 3-prong $\tauvis$ candidates. In both cases, the background rejection rate 
was found to be of $\mathcal{O}(10^2)$. All $\tauvis$ candidates were required to have $\pt>40~\GeV$ and $|\eta|<2.3$. 

\par Electron and muon reconstruction and identification is discussed in detail in Sections~\ref{sec:ele} and 
\ref{sec:mu}. Loosely identified and loosely isolated electrons were selected. To further ensure that the electrons originated 
from the primary vertex, $|z_0\sin\theta|<0.5$ and $|d_0/\sigma_{d_0}|<5$ we imposed.  
Electron candidates that satisfied these selection criteria were additionally required 
to have $\pt>20~\GeV$, and be within $|\eta|<2.47$, excluding transition between the barrel and end-cap regions $1.37<|\eta|<1.52$.
Muons were also loosely identified and isolated. In addition, impact parameter selection 
criteria were imposed on them to ensure that they originated from the primary 
vertex: $|z_0\sin\theta|<0.5$ and $|d_0/\sigma_{d_0}|<3$. 
Those that satisfied these requirements were required to have $\pt>20~\GeV$ and be within $|\eta|<2.5$.  

\par Overlap between the physics objects discussed above were resolved using the following priority list: 
$\mu\to e\to\tauvis\to\text{jet}$. If an electron was found within $\Delta R < 0.2$ 
of a reconstructed muon, the electron was removed from the list of physics objects in that event. This is 
because since electrons experience photon radiation and brehmsstrahlung at a higher rate than muons, they are   
more difficult to reconstruct. If a $\tauvis$ was reconstructed within $\Delta R<0.2$ of an electron 
or muon, the $\tauvis$ was removed. Likewise, if a jet was reconstructed within $\Delta R<0.2$ of an electron, muon 
or $\tau$ lepton, the jet was removed from the list.   

\par \met, the magnitude of the missing transverse momentum, was reconstructed after all 
the other physics objects were identified and overlaps were resolved. 
It was reconstructed as the negative vector sum of all the reconstructed and calibrated objects, and the reconstructed 
tracks that were not associated to any physics objects. All these objects were required to originate from the hard 
scatter event. This association with the hard scatter event reduced effects from pileup events on the reconstructed 
\met.

\par Table~\ref{tab:objcH} summarizes all the physics object selection criteria for this analysis.     

\begin{table}[!h]
\centering
  \resizebox{\textwidth}{!}{
   \begin{tabular}{|l|c|}
\hline
 Physics Object           								& Selection Criteria \\
\hline\hline
 \multirow{3}{*}{Jets}  & anti-$k_T$ with $\Delta R=0.4$ \\
			& $\pT>25~\GeV$, $|\eta|<2.5$    \\
			& JVT for $\pt<60~\GeV$ \\
\hline 
$b$-tagged Jets & $b$-tagging score at 70\% efficiency working point \\
\hline
  \multirow{3}{*}{\tauvis} & BDT score working point at 55\%(40\%) efficiency for 1(3)-prong for medium \\
 				& BDT score working point at 70\%(65\%) efficiency for 1(3)-prong for loose \\
				& $\pT>40~\GeV$, $|\eta|<2.3$ \\
\hline  
  \multirow{3}{*}{Electrons} & Loose ID \\
					& Loose Isolation \\
					& $\pT>20~\GeV$, $|\eta|<2.47$ excluding $1.37<|\eta|<1.52$ \\
\hline
  \multirow{3}{*}{Muons}     & Loose ID \\
					& Loose Isolation \\
					& $\pT>20~\GeV$, $|\eta|<2.5$ \\
\hline 
   \end{tabular}
}
\caption{Selection criteria for physics objects.}
\label{tab:objcH}
\end{table}
