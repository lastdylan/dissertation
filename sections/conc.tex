\chapter{Conclusion}
\par Since the discovery of a Higgs boson in 2012 by the ATLAS and CMS collaborations, 
efforts have been directed towards understanding the extent to which this Higgs boson 
represents the Standard Model Higgs boson. These efforts have included performing precision 
studies on the Higgs boson properties and comparing the results with those predicted 
by the Standard Model. The search for exclusive Higgs production described in this  
thesis was motivated by the desire to quantify the feasibility of performing Higgs precision 
studies through exclusive production, as it provides a clean environment with minimal 
background processes. Efforts after the Higgs boson discovery have also been directed towards 
searches for physics beyond the Standard Model. This thesis has described one such particular 
search -- the charged Higgs boson. Evidence for the charged Higgs boson would have 
been clear evidence for physics beyond the Standard Model.    

\par The search for exclusive Higgs boson production in LHC's Run I data was performed 
in the channel where the Higgs boson decays to a pair of \Wpm\ which subsequently decay 
leptonically. Decay of \Wpm\ to $\tau$ leptons were allowed only if the $\tau$ lepton 
decayed leptonically. A new method for separating exclusive events from inclusive events was devised, specifically for 
isolating exclusive Higgs events. This method was tested in both simulation and data, correcting any mis-modelling 
observed in simulation. Overall, background treatment was validated in two regions of data and observed to 
be reasonable. In the signal region, six events were observed in data while 3.00$\pm$0.78 events were expected from all 
the background processes. This discrepancy was evaluated to be 1.1$\sigma$ higher than expected, and hence 
not high enough to conclude the presence of exclusive Higgs events in the data-set. Upper limits were therefore 
set on the total production cross section of the exclusive Higgs boson. The set limits were 400 times 
the cross section predicted by the most popular model for the production of the exclusive Higgs boson.  

\par The search for the charged Higgs boson in LHC's Run II data was performed in the decay 
channel $H^+\to\tau\nu$. Observed candidate events in data were found to be 
consistent with Standard Model predictions, as opposed to the hMSSM model. 
Exclusion limits were set on the product of the production 
cross section of the charged Higgs boson and the branching ratio of its decay to a 
$\tau$ lepton and a $\nu_\tau$, for charged Higgs boson masses in the range 200$\to$2000~\GeV. 
In the hMSSM context, $\tan\beta=60$ values are excluded for the charged Higgs boson 
mass range $200\to 540~\GeV$. As the ATLAS detector keeps on collecting more data, 
results for this search keep on evolving. 	  
